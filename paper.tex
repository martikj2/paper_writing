\documentclass{iopart}
%\documentclass[pra,superscriptaddress,twocolumn,notitlepage,showpacs]{revtex4-1}

%\documentclass[aps,superscriptaddress,twocolumn,notitlepage,showpacs]{revtex4-1}
%\documentclass[aps,superscriptaddress,twocolumn]{nature}
\usepackage{graphicx} 
\usepackage{bm}% bold math2
%\usepackage{amsmath}
 
 
\usepackage{lipsum} % dummy text for the MWE
% Put % before of what you want disabled1
% Select what to do with todonotes: 
%\usepackage[disable]{todonotes} % notes not showed
\usepackage[draft]{todonotes}   % notes showed
% Select what to do with command \comment:  

 
\newcommand{\beq}{\begin{equation}}
\newcommand{\enq}{\end{equation}}
\newcommand{\bea}{\begin{eqnarray}}
\newcommand{\ena}{\end{eqnarray}}
\newcommand{\la}{\langle}
\newcommand{\ra}{\rangle}

\newcommand{\brho}{\bm{\rho}}
%\newcommand{\br}{\bm{r}}
\newcommand{\be}{\bm{e}}
\newcommand{\bn}{\bm{n}}
\newcommand{\bd}{\bm{d}}
\newcommand{\bbf}{\bm{f}}
\newcommand{\bx}{\bm{x}}
\newcommand{\by}{\bm{y}}
\newcommand{\bz}{\bm{z}}
\newcommand{\bk}{\bm{k}}
\newcommand{\bq}{\bm{q}}
\newcommand{\bp}{\bm{p}}
\newcommand{\bj}{\bm{j}}
\newcommand{\bu}{\bm{u}}
\newcommand{\bv}{\bm{v}}
\newcommand{\ba}{\bm{a}}
\newcommand{\bb}{\bm{b}}
\newcommand{\bh}{\bm{h}}
\newcommand{\bP}{\bm{P}}

\begin{document}
\title{Modelling lasing in plasmonic nano-particle arrays}
\author{J.-P. Martikainen$^1$, H. T. Rekola$^1$, T. K. Hakala$^1$ and P. T\"{o}rm\"{a}$^1$}
%\affiliation{COMP Centre of Excellence, Department of Applied Physics,
%Aalto University, P.O. Box 15100, Fi-00076 Aalto, Finland}
\address{$^1$ COMP Centre of Excellence, Department of Applied Physics,Aalto University, P.O. Box 15100, Fi-00076 Aalto, Finland}


\date{\today}

\begin{abstract}
We study lasing in 
arrays of metallic nanoparticles combined with active medium of dye molecules. We use a model which reduces the generally complicated spatial
and temporal dynamics of electromagnetic fields in the array structure
to a single ordinary differential equation for an oscillating dipole.
This approximation reduces the computational effort drastically and by
benchmarking our model against recent experiments we find satisfactory
agreement with experiments.
\end{abstract}
%\keywords{plasmonics,lasing}
\pacs{33.80.-b, 73.20.Mf,05.30.-d}
\maketitle 

%\ioptwocol

\section{Introduction}

\begin{itemize}
\item History of plasmon lasing?:
\item Why interesting:
\item Status of the experiments: 
\item Fast modulation speeds Refs 7-9 Odom
\item sub-wavelength mode confinement Refs 7-9 Odom
\item large losses poor directionality..tradeoff
\item Omat paperit ~\cite{martikainen_condensation_2014,shi_spatial_2014,vakevainen_plasmonic_2014}
\end{itemize}

Nanoscale light sources based on plasmons on metallic nano-particles
are of interest due to sub-wavelength modeconfinement as well as potentially fast modulation speeds. Basic tools of optics include amplifiers and lasers and such  devices would be needed in nano-optics as well.
A concept of a spaser i.e. surface plasmon amplification by stimulated emission of radiation was proposed by Bergman and Stockman~\cite{bergman_surface_2003}.
In a spaser electric fields due to surface plasmons close to a nano-particle
become amplified by stimulated emission due to inversion in the gain medium
surrounding the particle. Such prosesses have been experimentally 
observed~\cite{noginov_demonstration_2009,oulton_plasmon_2009}.

If there is an array of metallic nano-particles the collective response in the whole array can be quite different from that of individual nano-particles. In particular,
the array can support so-called surface lattice resonances (SLRs) which can be 
observed as very narrow extinction lineshapes. Such modes can furthermore
coherently and strongly couple to molecules surrounding the nano-particle array~\cite{vakevainen_plasmonic_2014,shi_spatial_2014}.
If gain medium feeds energy into de-localized SLR modes, it is possible that
for sufficiently inverted gain medium, gain can exceed the high loss rate
of a plasmonic system. This can take place easier for the SLR modes than for 
individual surface plasmons since the losses in a SLR mode can be weaker.
This was indeed observed in a recent experiment by Zhou {\it et al.}~\cite{zhou_lasing_2013}.


Our goal here is to present and benchmark a simple, but nevertheless
quite accurate way to model lasing in plasmonic nano-particle arrays.
Our method builds on neo-classical lasing theory, but takes the lasing
mode from the dipole fields of the metallic nano-particles. In the limit we consider, the complicated spatially dependent field dynamics can be replaced
by a single ordinary differential equation for a dipole.
This reduces the computational effort drastically from, for example, FDTD
simulations and enables rapid scans over broad range of parameters.
With benchmark our model against experimental results and find good agreement by adjusting only one uncertain parameter.


\section{Methods}
Previously plasmonic lasing has been approached using time dependent finite difference (FDTD)
methods for electromagnetic fields in conjunction with $4$-level description of dye molecules
forming the active material~\cite{dridi_model_2013}. Our approach shares some of the physical insights of
such models, but aims to lighter computational requirements and hence to a possibility
of scanning broad parameter ranges rapidly. We treat the gain material as a $4$-level system, but
in order to account for quantum yields that substantially below unity, we 
add an extra decay channel from the upper lasing level to the ground state. Furthermore, we simplify the
description of the electromagnetic field so that it is described by only a single ordinary differential equation
for the nano-particle dipole. In the next subsections we explain this in detail.


\subsection{Modelling nano-particle array and electric fields}
We show a schematic description of our system in Fig.~\ref{fig:schematic}.  We consider a square array
of golden ellipsoidal nano-particles bedded in a dye containing material
which constitutes the active medium of the lasing process. Dye molecules are
pumped with a laser pulse which is off-resonant with the SLR modes in the array. We will first explain how to describe the SLR modes and related
electric fields in the array.

\begin{figure} %[H]
\includegraphics[width=0.9\columnwidth]{schematic_modify.png}
\caption[Fig1]{Schematic description of the system we consider. There is a square array of
ellipsoidal (golden) nano-particles that are imbedded in a material containing dye molecules. Fields due to each nano-particle are
described by identical in-phase dipoles which are indicated in the figure by the blue arrows.
%The polarization of the pump laser is indicated with the arrow and is along the symmetry axis of the array.
}
\label{fig:schematic}
\end{figure} 

Oscillating electric field induces oscillating charge density in a metallic nano-particle. Often it is a good approximation to to describe induced fields
only in terms of an electric dipole. A coupled dipole approximation 
assumes that dipole in each nano-particle originates from the incoming driving field ${\bf E}_{in}({\bf r})$ and the dipole fields of all the other dipoles in the nano-particle array. Since these dipole fields depend on the electric dipoles
of other nano-particles, one must solve for a self-consistent solution
which transforms the problem into a solution of a set of linear equations.

If an electric field oscillating at angular frequency $\omega$ induces a dipole, polarizability $\alpha(\omega)$ is proportionality
constant between the field and the dipole  i.e. ${\bf p}=\bar{\alpha}(\omega) {\bf E}$.
Generally polarizability is a tensor since induced dipole does not have to be along the
inducing field and furthermore, material properties might be such that response is different
for fields in different directions. Our aim here is not to explore such properties and we will consistently assume that pump laser electric field is polarized along the array symmetry axis and that nano-particle ellipsoids
have a same main axis along $x$ and $y$-axis. For this reason it is enough
to treat $\alpha(\omega)$ as a scalar here.

In our modelling we assume that nano-particles are metallic ellipsoids
with main axis $a$, $b$, and $c$ along the $x$-, $y$-, and $z$-axis.
The material in nano-particles has permittivity $\epsilon_h$ while
the medium around nano-particles has $\epsilon_m$. 
Then the static polarizability~\cite{meier_enhanced_1983} of a single nano-particle is
\beq
\alpha_{stat}(\omega)=\frac{abc}{3}\frac{\epsilon_h-\epsilon_m}{N_x(\epsilon_h-\epsilon_m)+\epsilon_m}
\label{eq:astat}
\enq
Here $N_x$ is the depolarization factor along $x$-direction
\beq
N_x=\frac{abc}{2}\int_0^\infty \frac{dq}{(q+a^2)\sqrt{(a+q^2)(b+q^2)(c+q^2)}}.
\enq
(For ellipsoids static polarizability is in general different in different directions since main axes can be different. In our case $x$- and $y$-directions are the relevant once and we assume ellipsoids with $a=b$.)
In our computations we consistently assume gold nano-particles
and use tabulated values for the permittivities~\cite{johnson_optical_1972}.
The dielectric medium around the nano-particles is taken to have an index of refraction 
$n=1.52$ so that its permittivity is $\epsilon_m=\epsilon_0 n^2$.

To improve accuracy the expression~\ref{eq:astat} for the quasi-static polarizability should still be corrected by finite wavelength effects~\cite{meier_enhanced_1983}. This is
known as the modified long wavelength approaximation and the final single
particle polarizability we use is given by
\beq
\alpha_{MLWA}(\omega)=\frac{\alpha_{stat}(\omega)}{1-\frac{2i}{3}k^3\alpha_{stat}-\frac{k^2}{a}\alpha_{stat}},
\enq
where $k$ is the wavenumber in the medium. The term proportional to
$k^3$ is a correction due to radiation damping while the term proportional to
$k^2$ is due to dynamical depolarization~\cite{kelly_optical_2003}.

The previous discussion provides the basis of understanding 
the dipole response of a single ellipsoidal nano-particle. This
knowledge can now be used to gain understanding also of the response in 
nano-particle arrays.
The basic idea of the coupled dipole approximation (CDA) 
in an array of nano-particles is to assume that the field inducing a dipole in one of the nano-particles
in an array doesn't only arise from the incoming pump field but also
from the dipole fields of all the other
nano-particles in the array.~\cite{purcell_scattering_1973,zhao_extinction_2003,yurkin_discrete_2007}
CDA often provides an accurate description of many properties of
plasmonic arrays {\bf CITE SOMEBODY}~\cite{vakevainen_plasmonic_2014}. 

The electric field due to the oscillating dipoles ${\bf p}_j$ 
is given by
\begin{eqnarray}
&&{\bf E}_{DD}({\bf r})=\sum_{j} {\bf E}_{D,j}({\bf r})
= \\
&&\sum_{j} \frac{e^{i k R}}{4 \pi \epsilon_0} \left\{ 
k^2 \frac{\left( {\bf e}_r \times {\bf p}_j \right) \times {\bf e}_r }{R} + 
\bigl[ 
 3 {\bf e}_r \left( {\bf e}_r \cdot {\bf p}_j\right) - {\bf p}_j
\bigr] \left( \frac{1}{R^3} -\frac{i k}{R^2}\right)
\right\},\nonumber \label{DipoleFieldsA} 
\end{eqnarray}
where $\omega$ is the oscillation frequency, $j$ indicates the nano-particle at position ${\bf r}_j$, $\bm{e}_r=\bm{(r-r_j)}/R$,
and $R=|{\bf r}-{\bf r}_j|$.
Then the problem is to find all the dipoles ${\bf p}_i$ by solving the set of
linear equations
\beq
{\bf p}_i=\alpha(\omega)\left[{\bf E}_{in}({\bf r}_i)+{\bf E}_{DD}({\bf r}_i)\right].
\enq

In an infinite array with an incoming field normal to the nano-particle array we can take all dipoles to have a same magnitude
in which case the problem simplifies considerably. The effects of the array
structure can be summed into a term
\beq
S=\sum_{j\neq i}\frac{(1-ikr_{ij})(3\cos^2\theta_{ij}-1)e^{ikr_{ij}}}{r_{ij}^3}
+\frac{k^2\sin^2\theta_{ij}e^{ikr_{ij}}}{r_{ij}}
\enq
where we could choose $i$ to be, for example, the dipole in the center
of the array. $\theta_{ij}$ is the angle between the polarization of the 
incoming electric field and the vector
${\bf r}_{ij}={\bf r}_j-{\bf r}_i$ between two nano-particles. In terms
of $S$ the effective response of the whole array structure can be modelled
by the polarizability
\beq
\alpha_{eff}(\omega)=\frac{\alpha_{MLWA}(\omega)}{1-S\alpha_{MLWA}(\omega)}.
\enq
In SI units $\alpha_{eff}(\omega)$ should still be multiplied by $4\pi\epsilon_m$, where $\epsilon_m$ is the permittivity of the medium around
nano-particles.

In Fig.~\ref{fig:arraypolarizability} we show an example of the absolute value of the effective polarizability
in a square array with lattice spacing of $600\, nm$. This is the array structure we assume throughout this article.
 The broad peak at higher energies (lower wavelengths)
is due to SPP in single nano-particle while the narrow lower energy peak (at the $\Gamma$-point) is due to 
the surface lattice resonance. We expect lasing transition to occur at the frequency 
coresponding to this peak since there coupling between fields and nano-particles is strongest.
\begin{figure} %[H]
\includegraphics[width=0.9\columnwidth]{Polarizability_CDA.eps}
\caption[Fig3]{Effective polarizability for the square array of golden ellipsoidal nano-particles
as a function of wavelength.
 The main axis of the ellipsoids were $a=b=65\, nm$ and $c=37.5\, nm$ so that the volume of the particle is the same as the volume of the cylinders in~\cite{zhou_lasing_2013} and the lattice constant of the square array was $d=600\, nm$. For permittivities we used values from ref.~\cite{johnson_optical_1972}
}
\label{fig:arraypolarizability}
\end{figure} 


Since many properties the system response (such as extinction coefficient)
is well predicted by the CDA,
we might be able to describe also lasing reasonably well by taking the electric field
due to in-phase dipoles (as in the experiment by Zhou {\it et al.}~\cite{zhou_lasing_2013}) as the lasing field. In the limit where all nano-particles
are described by a same dipole, we can then describe the spatially dependent
electric field of the lasing mode with a single ordinary differential equation
for the dipole.

Note that in CDA one looks for a response to an infinitely long oscillating driving field.
In pulsed system this is an approximation since it reality, for example, retardation effects
will imply a delay before the CDA response is realized.
We model the dynamics of the dipole giving rise to the SLR mode with a damped harmonic oscillator model. Then the equation for the SLR mode dipole oscillating around the lasing transition frequency is given by
\beq
\frac{d^2p(t)}{dt^2}+\gamma_{SLR}\frac{dp(t)}{dt}+\omega_{SLR}^2p(t)=
\alpha(\omega)E_{in}(t)+
\frac{1}{\epsilon_m D_d}\frac{d^2P_{43}}{dt^2}.
\label{eq:dipoleeq}
\enq
Here $\alpha(\omega)$ is related to CDA effective polarizability via
\beq
\alpha(\omega)=\alpha_{eff}(\omega)\left[(\omega_{SLR}^2-\omega^2)-i\omega\gamma_{SLR}\right].
\enq
This choice ensures that for infinitely long driving field (with no active medium)
$E_{0}\exp(-i\omega t)$ the response approaches 
$p=\alpha_{CDA}(\omega)E_{0}$. $\gamma_{np}$ describes losses
of the SLR mode and from the width of the response peak
we estimate this as $\gamma_{SLR}=10\, meV$.
$\omega_{SLR}$ is the SLR mode frequency which in turn is the energy
difference between upper and lower lasing levels in the active medium.

The dipole is driven by the polarization density $P_{43}(t)$ of the 
lasing transition. From now on we make the assumption that nano-particle dipoles are only driven by the active medium
and not by the input laser pulse which only excited molecules in the active medium. In this case $\alpha(\omega)E_{in}(t)$ term on the right hand side
of Eq.~(\ref{eq:dipoleeq}) can be ignored.
This is reasonable if plasmonic modes are off-resonant with the pump
frequency or if they decay so quickly as to have only small impact on the dynamics. (In Eq.~(\ref{eq:dipoleeq})
$D_d$ which appears in the denominator is a proportionality constant
between dipole magnitude and the electric field due to dipoles in the array structure. Read more on this in the next subsection.)



\subsection{Lasing model}
In our system there is (possibly) a strong pump laser exciting
dye molecules. Pumped molecules decay quickly among the excited states
into a long lived state that is the excited state of the lasing transition.
Due to its long life time strong inversion density can develop
between this state and the (quickly decaying) excited state close to the ground state. In these
conditions stimulated emission processes can become important as a strong lasing field develops in the nano-particle array. These basic principles
suggest a minimal $4$-level description of the active media along the
neo-classical lasing theory~\cite{anthony_e._siegman_lasers_1986}.
In the following we do not assume weak coupling between emitters and the fields
so we have, for example, the possibility of Rabi oscillations
included in the formalism.
For the active media we have then following set of equations. 
In Fig.~\ref{fig:4level_schematic} we show a schematic description
of the dye molecules of the active medium.
\begin{figure} %[H]
\includegraphics[width=0.9\columnwidth]{4level_schematic.pdf}
\caption[Fig3]{$4$-level description of the dye molecules together with the notation
we use. Lasing transition is indicated by the thick blue arrow and the pump with the black arrow. In these transition there can be a strong
electric field present and stimulated emission can be important.
Spontaneous emission processes are indicated by dashed arrows. We have also
added (green arrow) a separate decay channel away from the
upper lasing level to the ground state to capture some of the effects from
non-unit quantum yield. If quantum yield $\eta$ is one, this transition has infinite lifetime and otherwise $\tau_{13}=\tau_{23} \eta/(1-\eta)$. In our 
examples we choose $\eta=0.16$.
}
\label{fig:4level_schematic}
\end{figure} 


There are equations for polarization densities for transition $1-4$ pumped by the laser
and for the lasing transition $2-3$:
\beq
\frac{d^2P_{14}(t)}{dt^2}+\Delta\omega_{14}\frac{dP_{14}(t)}{dt}+\omega^2P_{14}(t)=
\frac{3\omega_{14}\epsilon_m\lambda_{14}^3\gamma_{rad,14}}{4\pi^2}\Delta N_{14} E_{in}(t),
\enq
\beq
\frac{d^2P_{23}(t)}{dt^2}+\Delta\omega_{23}\frac{dP_{23}(t)}{dt}+\omega_a^2P_{23}(t)=
\frac{3\omega_{23}\epsilon_m\lambda_{23}^3\gamma_{rad,23}}{4\pi^2}\Delta N_{23} E_{DD}(t)
\enq
where $E_{DD}(t)$ describes the field due to oscillating dipoles in the nano-particle array at frequency $\omega_{23}=\omega_3-\omega_2$. 
While $E_{in}(t)$ is the electric field of the pump pulse. We take the pump pulse to be a gaussian pulse
\beq
E_{in}(t)=E_0\sin(\omega_{14}t)e^{-\frac{(t-3\tau)^2}{\tau^2}},
\enq
where $\tau$ is the pulse length and $E_0$ is related to the areal energy
density $\cal{E}$ of the pump pulse through
\beq
E_0=\sqrt{\frac{2{\cal E}\sqrt{2/\pi}}{\epsilon_0 c_0 n\tau}}
\enq
with $\epsilon_0$ permittivity of the vacuum and $c_0$ the vacuum speed of light.

$\Delta N_{14}=N_4-N_1$ and  $\Delta N_{23}=N_3-N_2$ are inversion densities for the transitions. $\lambda_{23}$ and $\lambda_{14}$ are transition
wavelengths while $\gamma_{rad,14}$ and $\gamma_{rad,23}$ are the radiative
lifetimes for the transitions. Decay coefficients
$\Delta\omega_{23}$ and $\Delta\omega_{14}$ are larger radiative decay rates
since they include also dephasing processes due to, for example, collisions.
%The decay coefficients $\gamma_{rad,14}$
%and $\gamma_{rad,23}$ are larger than for single molecules since they
%also contain effects due collisional broadening in the gain material.


Finally the ``cavity'' electric field is taken to be the field originating from all dipoles oscillating in-phase in the nano-particles. In principle,
this field has a pronounced position dependence, but for simplicity we
average this field in the array to get a characteristic field strength
and the proportionality constant between this field and the nano-particle dipole. More specifically we use
\beq
{\bf E}_{DD}(t)=D_d p(t)\exp(-i\omega_{23} t),
\enq
where the proportionality constant 
\beq
D_d=\sqrt{\langle |{\bf E}_{DD}({\bf r})|^2\rangle }/p_0.
\label{eq:DDaverage}
\enq
${\bf E}_{DD}({\bf r})$ is the combined field of the in-phase dipoles and
${\bf p}_0=\alpha_{eff}(\omega_{23}){\bf E}_{in}$. (Strength of the input field
${\bf E}_{in}$ in this formula is irrelevant since it cancels
away when computing the proportionality
constant $D_d$.) The average is taken over all locations in the structure except points inside the nano-particles.

In Fig.~\ref{fig:DDaveraging} we demonstrate that the proportionality factor
between the field and the dipole is in fact dominated by the fields close to
the nano-particles. More than half of its value originates from the average that only includes
the volume within $50\, nm$ of the nano-particles. 
It should be noticed that our averaging procedure takes
to some extent into account the enhancement of the 
stimulated emission rate (which is proportional to intensity)
close to the dipoles. On the other hand decay enhancement 
due to Purcell effect is a separate phenomena. Purcell factor 
can be measured independently by time resolved measurement
of molecular decay~\cite{oulton_plasmon_2009,zhou_lasing_2013,akselrod_probing_2014}.

\begin{figure} %[H]
\includegraphics[width=0.7\columnwidth]{Array_Field_Average_Volume.pdf}
\caption[Fig1]{How the average of the electric field strength squared depends on the volume over which the average is calculated? The light gray area in the inset demonstrates the area around each nano-particle
that is to included in the average. Result is compared to the one where averaging area was not restricted
other than by excluding points inside the nano-particles. That average
is denoted by $\langle |{\bf E}|^2\rangle$ on the y-axis.
We assumed similar parameters for the nano-particles as in the experiments~\cite{zhou_lasing_2013},
but approximated the gold particles as ellipsoids instead of cylinders. 
The main axis of the ellipsoids were $a=b=130\, nm$ and $c=37.5\, nm$ so that the volume of the particle is the same as the volume of the cylinders in Ref.~\cite{zhou_lasing_2013}. The lattice constant of the square array was $d=600\, nm$.
}
\label{fig:DDaveraging}
\end{figure} 


In order to get a closed set of equations 
we still need equations for the level densities
of all $4$-levels. Since it is not clear what fraction
of the dye molecules are actually available to participate
in the lasing at $\Gamma$-point we adjust the molecule
concentration $N=N_1+N_2+N_3+N_4$ from the physical concentration $C$
such that $N=fC$, where $f$ is the fraction of active molecules
and should be somewhere between $0$ and $1$.
%(${\bf e}_{in}$ is the unit vector in the direction of
%the polarization in the gain media..presumably can be taken as along $x$ for %the stimulated
%processes at least.)
\beq
\frac{dN_{1}(t)}{dt}=\frac{N_4}{\tau_{14}}+\frac{N_2}{\tau_{12}}+\frac{N_3}{\tau_{13}}-\frac{1}{\hbar\omega}
Re\left[\frac{dP_{14}(t)}{dt} E_{in}(t)\right]
\enq
\beq
\frac{dN_{4}(t)}{dt}=-\frac{N_4}{\tau_{14}}-\frac{N_4}{\tau_{43}}+\frac{1}{\hbar\omega}
Re\left[\frac{dP_{14}(t)}{dt} E_{in}(t)\right]
\enq
\beq
\frac{dN_{3}(t)}{dt}=\frac{N_4}{\tau_{43}}-\frac{N_3}{\tau_{23}}-\frac{N_3}{\tau_{13}}+\frac{1}{\hbar\omega_a}
Re\left[\frac{dP_{23}(t)}{dt} E_{DD}(t)\right]
\enq
\beq
\frac{dN_{2}(t)}{dt}=\frac{N_3}{\tau_{23}}-\frac{N_2}{\tau_{12}}-\frac{1}{\hbar\omega_a}
Re\left[\frac{dP_{23}(t)}{dt}E_{DD}(t)\right].
\enq
There is no need for terms accounting for stimulated emission
in $4-3$ and $1-2$ transitions since they are assumed to decay so quickly
that spontaneous emission remains dominant.

Close to metallic nano-particles Purcell effect can be 
substantial which means that molecular decay rates close to nano-particles
can be very different from those further away. We adjust the life time
of the $2-3$ lasing transition with a Purcell factor $F$ so that
$T_{23}\rightarrow T_{23}/F$ and $\gamma_{rad,23}\rightarrow \gamma_{rad,23}F$. Typically we have chosen the experimentally measured Purcell factor of $F=200$~\cite{zhou_lasing_2013}, but this is expected
to vary between systems and is often uncertain without separate
measurement.

For concreteness we use here similar parameters as have been used to
describe the experiments by Zhou {\it et al.}~\cite{zhou_lasing_2013} where IR-140 dye molecule
was used. In particular we choose $\tau_{34}=\tau_{12}=10\, fs$
and $\tau_{23}=\tau_{14}=1\, ns$. Pump wavelength is $\lambda_{14}=800\, nm$,
index of refraction in the active media $n=1.52$, and
absorption cross sections for the pump and lasing transitions are
estimated as $\sigma_{14}=3.6\cdot 10^{16}\, {\rm cm^2}$ and
$\sigma_{23}=4.5\cdot 10^{16}\, {\rm cm^2}$~\cite{sperber_experimental_1988}. 
The damping coefficients due to finite linewidths were chosen
as $\Delta \omega_{14}=2.94\cdot 10^{14}\, 1/s$ 
and $\Delta \omega_{23}=2.49\cdot 10^{14}\, 1/s$. Dye concentration
was $C=2\cdot 10^{18}, 1/cm^3$ unless otherwise mentioned.

Note that in the above equations we also added an extra decay channel from the upper lasing level directly to the ground state to account for
non-radiative transitions that reduce the quantum yield $\eta$. We have in particular
$\tau_{nr}=\tau_{32} \eta/(1-\eta)$. We took the quantum yield for the IR-140 molecules to be $16\%$ as was estimated in the experiments by Zhou {\it et al.}~\cite{zhou_lasing_2013}. 




\section{Results}

\begin{itemize}
\item Output power and FWHM as a function of input pulse properties
\item FWHM should be reduced as concentration grows according to Odom people.: Close to threshold FWHM is smaller with larger concentration
high above threshold FWHM is limited by the stimulated emission rate
which is higher for larger concentrations. Then FWHM is larger for larger'
concentrations.
\item Does the result depend a lot on the quantum yield term?  No...it doesn't. 
\item How does the threshold/output change with nano-particle volume?
\end{itemize}

In our model most parameters are either from the literature or
experiments. The most uncertain parameters would typically be the fraction
of the active molecules and the Purcell factor for different transitions.
The parameters we use here are similar to the experiment by~\cite{zhou_lasing_2013} which found a lasing threshold
at $0.23\, mJ/cm^2$. We demonstrate how our prediction
for the lasing threshold depends on the fraction
of active molecules in Fig.~\ref{fig:threshold}. Our results
match experimental results well if $f\approx 0.12$.
If all molecules are
available then the threshold is lower.

\begin{figure} %[H]
\includegraphics[width=0.45\columnwidth]{Threshold_fraction.eps}
\caption[Fig4]{Threshold as a fraction of active molecules when the 
Purcell factor was taken as $F=200$. Parameters
were similar as in the experiment by Zhou {\it et al.}~\cite{zhou_lasing_2013}. With $f\approx 0.12$ the result matches the experimental results.}
\label{fig:threshold}
\end{figure} 

Having found all the necessary parameters, 
in Fig.\ref{fig:demosimu} we demonstrate the behavior
of lasing pulse energy as a function of pump pulse energy.
Radiated energy is proportional to $\int |p(t)|^2dt$ and we scale
this signal so that its maximum value is $1$ in the figure. As is clear there is a clear threshold after which the lasing pulse energy grows in a non-linear fashion.
This is consistent with the FDTD calculation in by Dridi and Schatz~\cite{dridi_model_2013}.
 
The Fig.\ref{fig:demosimu} also shows the FWHM of the
Fourier-transform of dipole signal $p(t)$. For a weak pump this approaches the loss rate of the SLR $\gamma_{SLR}$. However, since in this regime the energy in the lasing mode is very small
this might be hard to observe. 
As the pump becomes larger FWHM is reduced by about factor of $3$
so that just above threshold the FWHM is about $2.2\, nm$ with a $40\, fs$ pump pulse
and about $1.8\, nm$ with a $160\, fs$ pulse. These results are close to experimentally observed 
values. The minimum FWHM is reduced if the pump pulse is longer demonstrating that it is a 
quantity that does depend on the details of the pump pulse even if, for example, 
the  lasing threshold can be quite insensitive to the length of the pump pulse. In this parameter regime the
energy in the lasing pulse increases  with pump pulse length and was 
around $20$ times larger in the simulation with a longer pump pulse.
As the pump increases further, FWHM starts to increase. This increase is related to
shortening of the lasing pulse in time or, alternatively, to increasing maximum stimulated emission
rate into the lasing mode. 

Since Purcell factor and fraction of active molecules can be quite uncertain parameters it is useful to observe
that threshold remains essentially the same if the product $F\times f$ remains the same.
If, for example, we choose $F=24$ and $f=1$, the threshold is essentially
the same as in
Fig.\ref{fig:demosimu}. However, then the energy of the lasing pulse can be higher since
active media has more energy available for lasing.



\begin{figure} %[H]
\includegraphics[width=0.45\columnwidth]{Odom_demo_a.eps}
\includegraphics[width=0.45\columnwidth]{Odom_demo_b.eps}
\caption[Fig5]{Energy in the lasing mode and the FWHM of the lasing signal.
We choose the Purcell factor for the lasing transition to be $F=200$, the fraction of dye molecules
participating in the lasing process as $f=0.05$, and $\gamma_{np}=10\, meV$.
The concentration of dye molecules was set at $c=2\cdot 10^{18}\, 1/cm^3$.
In (b) we show results with two different pump pulse lengths -- $40\, fs$ with a solid blue line
and $160\, fs$ with a dashed red line.}
\label{fig:demosimu}
\end{figure} 


Generally, the threshold is moved higher if the plasmon losses increase or 
if the molecule concentration is reduced. The effect of non-unit
quantum yield was small in the parameter regime we explored.
In the vicinity of the threshold FWHM of the lasing pulse is also lower for higher molecule concentrations.
Well above threshold this is reversed and higher concentrations imply 
have somewhat higher FWHM because the system has then higher stimulated
emission rate which shortens the lasing pulse. We demonstrate this in 
Fig.~\ref{fig:demosimu_2concentrations} by simulations with two different molecule concentrations.
\begin{figure} %[H]
%\includegraphics[width=0.45\columnwidth]{Odom_concentration_a.eps}
\includegraphics[width=0.45\columnwidth]{Odom_concentration_b.eps}
\caption[Fig6]{The FWHM of the lasing signal for two different molecule
concentrations.
We set the fraction of dye molecules
participating in the lasing process as $f=0.09$ and the
concentration of dye molecules was either $C=2\cdot 10^{18}\, 1/cm^3$ (solid line)
or $C=4\cdot 10^{18}\, 1/cm^3$ (dashed line).
}
\label{fig:demosimu_2concentrations}
\end{figure} 



%\section{Discussion}



\section{Conclusions}
We have developed a computationally light method to study lasing in plasmonic nano-particle arrays. By benchmarking the model against
experimental results,
we demonstrated that despite simplifying assumptions the model reproduces many experimental
observations well. 
Considering the approximations we made and uncertainties in the parameters we consider
our model surprisingly accurate. It reproduces most features of the full FDTD calculations, but
with far less computational effort. This can prove to be useful if large parameter
regimes must be scanned or in gaining a quick overview of the expected properties of some specific plasmonic structure.
Here we applied the model only to a fairly simple system of square array, but our approach generalizes also to other situations for which CDA
is accurate.



%\begin{acknowledgments}
\ack{
This work was supported by the Academy of Finland through its
Centres of Excellence Programme (Projects No. 251748, No. 263347 and No. 13272490) and by the
European Research Council (ERC-2013-AdG-340748- CODE)}
%\end{acknowledgments}

\vspace{1cm}

%\bibliographystyle{apsrev}
\bibliographystyle{iopart-num}
\bibliography{./bibli_Plasmon_lasing}

\end{document}
